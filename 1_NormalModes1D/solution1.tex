\documentclass{article}

\usepackage{color}
\usepackage{amssymb}
\usepackage{mathrsfs}
\usepackage{graphicx} % Required for the inclusion of images

\setlength\parindent{0pt} % Removes all indentation from paragraphs
\renewcommand\footnoterule{\rule{\linewidth}{0.4pt}}


\begin{document}
\title{Physics Problems\\ Solution} % Title

\author{For Pecouzaustralianovsky} % Author name

\date{\today}
\maketitle

\section*{Normal Modes in 1D\footnote{A. P. French. Vibrations and Waves. 2003}}

\textbf{a. Discrete}\\

\textit{Consider a system of N particles which is subject to a restoring force proportional to the separation to the equilibrium position ( $ F \quad \alpha \quad x $ ). How many normal modes has this system?}

\begin{center}
\noindent\rule{3cm}{0.4pt}\\
\end{center}

\vspace{2mm}

N normal modes.\\


\textbf{b. Continuous }\\

\textit{Consider now a continuous system of particles confined to a length $L$ and fixed borders. Which are normal wave lengths ($\lambda_n$) and the normal frequencies ($\nu_n$) in terms of $\nu$(the speed of waves in the medium)? Is it correct to think that this system has infinite normal modes?. If it does not have infinite normal modes, give a order of magnitude of the number of normal modes.}

\begin{center}
\noindent\rule{3cm}{0.4pt}\\
\end{center}

\vspace{2mm}

The normal wave lengths and frequencies are given by (1) and (2). The system does not have infinite normal modes, however the number of normal modes its big enough to consider it as infinite. The number of normal modes is in the order given by (3), where $a$ is the average separation between particles in the system. \\

\begin{equation}
\lambda_n = \frac{2L}{n}
\end{equation}

\begin{equation}
\nu_n = \frac{n \nu}{2L} 
\end{equation}

\begin{equation}
N = \frac{L}{a}
\end{equation}


\end{document}