\documentclass{article}

\usepackage{amssymb}
\usepackage{mathrsfs}
\usepackage{graphicx} % Required for the inclusion of images

\setlength\parindent{0pt} % Removes all indentation from paragraphs
\renewcommand\footnoterule{\rule{\linewidth}{0.4pt}}

\begin{document}
\title{Physics Problems\\ Problem \#1}

\author{For Pecouzaustralianovsky}

\date{December 7, 2014}
\maketitle


\section*{Normal Modes in 1D\footnote{A. P. French. Vibrations and Waves. 2003}}

\textbf{a. Discrete}\\

Consider a system of N particles which is subject to a restoring force proportional to the separation to the equilibrium position ( $F \quad \alpha \quad x$ ). How many normal modes has this system?\\

\textbf{b. Continuous }\\

Consider now a continuous system of particles confined to a length $L$ and fixed borders. Which are normal wave lengths ($\lambda_n$) and the normal frequencies ($\nu_n$) in terms of $\nu$(the speed of waves in the medium)? Is it correct to think that this system has infinite normal modes?. If it does not have infinite normal modes, give a order of magnitude of the number of normal modes.


\end{document}